\documentclass[12pt]{article}
\usepackage{overcite}
\usepackage{fullpage}
\usepackage{amsmath}
\renewcommand{\familydefault}{\sfdefault}
\renewcommand{\thefootnote}{\textit{\alph{footnote}}}

\usepackage{amsmath}
\usepackage{multirow}
\usepackage{xspace}
\usepackage{booktabs}
\usepackage{float}
\usepackage{subfig}
\usepackage{textcomp}
\usepackage{color}
\usepackage{hyperref}
\newcommand{\textapprox}{\raisebox{0.5ex}{\texttildelow}}

\newcommand{\qcaN}{QCArchive}
\newcommand{\qcelN}{QCElemental}
\newcommand{\qcskN}{QCSchema}
\newcommand{\qcngN}{QCEngine}
\newcommand{\qcfN}{QCFractal}
\newcommand{\qcptlN}{QCPortal}

\newcommand{\qca}{{\sc{\qcaN}}\xspace}%
\newcommand{\qcai}{{\sc{\qcaN\xspace Infrastructure}}\xspace}%
\newcommand{\qcsk}{{\sc{\qcskN}}\xspace}%
\newcommand{\qcel}{{\sc{\qcelN}}\xspace}%
\newcommand{\qcng}{{\textsc{\qcngN}}\xspace}%
\newcommand{\qcf}{{\sc{\qcfN}}\xspace}%
\newcommand{\qcptl}{{\sc{\qcptlN}}\xspace}%
\newcommand{\mqcas}{{\sc{mqcas}}\xspace}%

\parskip 10pt

\begin{document}

\noindent
\textbf{Reviewer 1:}

The manuscript documents an ongoing effort to establish a computational
chemistry resource that provides both extensive data and associated tools for 
computation and analysis.  Their open-source and community-driven approach is
commendable, and the manuscript is publishable after the following points are
taken into consideration.

1) The objectives of the project are ambitious and no doubt many previous
attempts have floundered due to problems arising during their development.
Despite this, there is little discussion on the challenges faced in the 
QCArchive project, what possible solutions were considered, and the reasoning
behind the resolutions of these challenges.  For example, can data from
multiple programs be combined into one data set and, if so, how are the various
program options (the defaults of which can vary widely across packages) treated?
When can levels of theory from two different packages be considered equivalent?
(SCF convergence thresholds, algorithms, DFT grids etc.)  How are excited
states handled?

\noindent \underline{Author Response:} \textit{Data from multiple programs can
indeed be combined into one data set, and we have added the following
statement on p.12 to indicate this:}

\noindent \textcolor{red}{"For example, unique identifiers for given single
point computation are molecule, method, basis, program, and keyword arguments
create a unique computation."}

\noindent
\textit{Data is stored/retrieved/displayed based
on the unique identifiers given above. Trying to reconcile the 'same'
calculation using different software is beyond the scope of the project, and
is one area where previous projects may have stumbled. Excited states are not
currently handled, but can easily be added.}

2) I downloaded the software and, although I was able to run the examples in the 
manuscript, it felt very much like I had been given a few fish and not taught
how to fish.  Perhaps the authors could include some examples of how to access
help topics, and how to obtain listings of valid keywords and data identifiers.
Without these the experience feels like groping around in the dark.

\noindent \underline{Author Response:} \textit{While continued usability of
the package is one of our highest short-term priorities, in the meantime we
have provided examples to help users get started here:}

\begin{center}
\url{https://qcarchive.molssi.org/examples/}
\end{center}

3) An important aspect of any collection of data such as the MQCAS is data
integrity.  Provenance meta-data are included, which would allow for bulk data
invalidation if a bug were to be discovered, but what about other potential
problems such as SCF convergence to a saddle point?  Figure 1.  indicates a
review process is carried out on contributions to the database, but no details
are given.  I think for a topic which is so central to the project, more
discussion is required.

\noindent
\underline{Author Response:} \textit{We agree fully with the reviewer that quantum
chemical calculations are complex and frequently lead to unsought results.
Thus, we have recently employed a review process by the \mqcas maintainers for
data submitted sets.  We have added the following on p.5:}

\noindent \textcolor{red}{``Submitted datasets are reviewed by the \mqcas
maintainers for integrity and often require comparisons to literature values
to ensure consistency (where possible and appropriate) before accepting a
contribution.''}

4) In Section 3.2 four software components are mentioned (page 7, line 25), yet 
five are summarised in the subsequent bullet points.  Furthermore, only four of
of the five are expanded on in the following sub-subsections (QCSchema is not 
covered).

\noindent \underline{Author Response:} \textit{We have corrected the number noted on
p. 7 from ``four'' to ``five''.  However, we have not provided further
elaboration on \qcsk in this manuscript because it involves a much larger
number of contributors than the current author list and its complexity requires
a separate article to be published later.  To clarify this in the paper, we have
added the following parenthetical statement to the \qcsk bullet on p.8.:}

\noindent \textcolor{red}{``(This important component, which has been
developed somewhat separately from the rest of the \qca package, will be
decscribed in a later publication.)''}

5) It would be interesting to know if there are any efforts being made toward
integrating other chemical databases such as ChemSpider and whether or not 
an HTTP API exists or is being considered.

\noindent \underline{Author Response:} \textit{We have had several
conversations with PubChem and Citrine about cross referencing data, but this
has not yet been complete, and thus we feel it would be premature to comment
on this in the manuscript.  \qcfN~interacts exclusively over standard
TCP/IP (HTTP/REST) protocols that can be accessed from any language as
discussed briefly in Fig 4.}

\noindent\makebox[\linewidth]{\rule{\linewidth}{0.4pt}}

\noindent
\textbf{Reviewer 2:}

\noindent
The authors provide an overview of a new archive for quantum-chemical data that is accessible for free for users. It offers a large variety of options, interfaces with many quantum-chemical programs and shows huge potential for further extension. This is without doubt a great initiative. The article is well written and easy to follow and has the potential for a large number of citations as soon as more users have become aware of this initiative. 

\noindent
The article can almost be published as is and I would just like to mention four points for a very minor revision.

1) Ultimately, it is up to the editor to decide on the article type, but as this article does not provide an in-depth review of a methodology published in the literature but revolves more around software for data mining, I feel that a more appropriate article type would be “Software Focus”.

\noindent \underline{Author Response:} \textit{This is clearly up to the
editor, but we have no preference.}

2) Could the authors comment on their strategies to ensure that existing databases are monitored for updates? For instance, they use reaction databases such as ACONF, S22, etc. and in the past the reference values for such sets have been updated multiple times. In addition to having access to structures and low-level energies, it is also important for the CMS community to keep track of the currently most accurate reference data.

\noindent \underline{Author Response:} \textit{We agree, and we make a
concerted effort to keep up to date on the literature and update the values
where available and have also created the open gateway to adding new values.
An example of this is the S22 dataset where currently the original benchmark
values as well as the ``S22a'' and ``S22b'' have been added to the dataset.}

3) Ref. 63 is not the correct reference for the B3LYP functional. 

\noindent
\underline{Author Response:} \textit{We have removed the original Ref.\ 63 and
added the following two corrrect references (recognizing, of course, that not
all quantum chemical programs define this functional equivalently.)}

\noindent
\textcolor{red}{(66) Becke, A. D. Density-functional thermochemistry. III. The
role of exact exchange. J. Chem. Phys. 1993, 98, 5648–5652.}

\noindent
\textcolor{red}{(67) Lee, C.; Yang, W.; Parr, R. G. Development of the Colle-Salvetti
correlation-energy formula into a functional of the electron density. Phys. Rev.
B. 1988, 37, 785–789.}

4) It is common to write ``def2-SVP'' instead of ``def2-svp''.

\noindent \underline{Author Response:} \textit{All instances have of
``def2-svp'' have been changed to ``def2-SVP''.}

\noindent\makebox[\linewidth]{\rule{\linewidth}{0.4pt}}

\noindent
\textbf{Reviewer 3:}

\noindent The review by Smith et al. describes the QCArchive project. This is
an initiative by The Molecular Sciences Software Institute (MolSSI). QCArchive
is an important resource that aims to organize, curate, and share quantum
chemistry data for the benefit of the entire computational molecular sciences
community.  QCArchive consists of multiple fully open sources of SW modules
that control specific functionalities. All are accessible via GitHub. I think
this is an extremely useful review that gives an introduction about the basic
functionality of QCArchive and provides short snippets of code and how to use
it. This manuscript is well written and overall as a great fit for WIRES CMS.

My only minor comments would be:

1.	For the benefit of users and readers, it would be nice is to supplement code snippets with links to the entire example/tutorial in the jupyter notebook.

\noindent
\underline{Author Response:} \textit{We agree, and we have added to p.19 the 
following text, which includes a link to a tutorial (as Ref. 69):}

\noindent
\textcolor{red}{``A \qcf quick start guide to reproduce the above example can be
found on our website.''}

2.	Is QCArchive unique? Are there any comparable efforts in the EU or perhaps in China? It would be great to complement conclusions with comparison (if any).

\noindent
\underline{Author Response:} \textit{ QCArchive is unique in the sense that the
data it stores are subject to rigorous standards of units, representation, etc.
However, there are certainly other existing efforts with overlapping goals, such
as NOMAD and the Materials Project mentioned in the manuscript's introduction.
However, to provide further context, we have added the following parenthetical
with references
to the introduction:}

\noindent
\textcolor{red}{``(e.g., ioChem-BD, the Pitt Quantum Repository, and the
Computational Chemistry Comparison and Benchmark DataBase)''}

\noindent
\textbf{Other Changes:}

\noindent \textit{We have updated Figure 10 to provide a more complete visualization of the application.}

\end{document}
